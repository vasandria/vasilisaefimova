% "Лабораторная работа 10"

\documentclass[a4paper,12pt]{article} % тип документа

% report, book

%  Русский язык

\usepackage[T2A]{fontenc}			% кодировка
\usepackage[utf8]{inputenc}			% кодировка исходного текста
\usepackage[english,russian]{babel}	% локализация и переносы


% Математика
\usepackage{amsmath,amsfonts,amssymb,amsthm,mathtools} 


\usepackage{wasysym}
\usepackage{hyperref}


%Заговолок
\author{Ефимова В.С. ИВТ3}
\title{Аннотированный список ресурсов интернета, содержащих рекомендации по работе в \LaTeX}
\date{\today}


\begin{document} % начало документа
\maketitle
\newpage
\begin{center}
{\Large Аннотированный список ресурсов интернета, содержащих рекомендации по работе в \LaTeX}
\end{center}
\begin{enumerate}
\item \href{https://www.ibm.com/developerworks/ru/library/latex_tutorial_01/index.html}{\textbf{Евгений Балдин. Работа в \LaTeX . Создание документа на примере подготовки курсовой работы}}. В данной статье рассмотрен процесс создания \LaTeX -документа на примере подготовки курсовой работы.
\item \href{https://www.intuit.ru/sites/default/files/documents/needhelp/LaTeX.pdf}{\textbf{С. М. Львовский. Набор и верстка в системе \LaTeX}}. Настоящее пособие посвящено популярной издательской системе,
основанной на TEX’е, — системе LATEX. Оно пригодится как читателю, которому необходимо по роду своей работы готовить тексты с формулами,
так и специалисту по компьютерной верстке.
\item \href{https://losst.ru/kak-polzovatsya-latex}{\textbf{Admin сайта losst.ru. Как пользоваться \LaTeX}}. В этой статье рассмотрено, как пользоваться LaTex для начинающих, тех, кто только решил познакомиться с этим языком. Здесь разобраны его основные возможности и приведены несколько примеров. Работа выполнена на основе редактора LaTeXila, который можно считать наиболее простым для начинающих по мнению автора.
\item \href{https://pikabu.ru/story/latex_dlya_novichkov_vvedenie_4999494}{\textbf{Пользователь madarexxx сайта picabu.ru. \LaTeX для новичков}}. Серия постов предназначеная для абитуриентов и студентов -- тем кому нужно писать большое количество текста за минимальное время.
\item \href{https://wch.github.io/latexsheet/}{\textbf{Шапргалка по \LaTeX}}. Небольшая шпаргалка содержит минимум информации, позволяющий оформлять простые документы, такие, как резюме, отчет, контрольная работа и тому подобное, используя макропакет компьютерной верстки в \LaTeX
\item \href{https://ru.wikibooks.org/wiki/LaTeX}{\textbf{Викиучебник по \LaTeX}}. Это учебник по языку \TeX, её расширениям (\LaTeX, XeLaTeX, и др.), дополнительным инструментам (pdflatex, bibtex и др.) и некоторым пакетам.
\item \href{http://herba.msu.ru/shipunov/software/tex/catlatex.pdf}{\textbf{С.В. Клименко, М.В. Лисина. \LaTeX и его команды}}. Полный подробный перечнень всех команд в \LaTeX .

\end{enumerate}  
\end{document} % конец документа