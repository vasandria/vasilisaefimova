% "Лабораторная работа"

\documentclass[a4paper,12pt]{article} % тип документа

% report, book

%  Русский язык

\usepackage[T2A]{fontenc}			% кодировка
\usepackage[utf8]{inputenc}			% кодировка исходного текста
\usepackage[english,russian]{babel}	% локализация и переносы


% Математика
\usepackage{amsmath,amsfonts,amssymb,amsthm,mathtools} 


\usepackage{wasysym}

%Заговолок
\author{Ефимова В.С., ИВТ3}
\title{Основы работы в \LaTeX{}}
\date{\today}


\begin{document} % начало документа
\maketitle
\newpage
\section{Издательские системы}
\textbf{Настольные издательские системы (НИС)} — это программы, предназначенные для профессиональной издательской деятельности, позволяющие осуществлять электронную верстку широкого спектра основных типов документов. НИС используются для подготовки буклетов, оформления журналов и книг предназначены специальные издательские системы. Они позволяют готовить их и печатать на принтерах или выводить на фотонаборные автоматы сложные документы высокого качества.

Предусмотренные в программных пакетах данного типа средства позволяют:
\begin{itemize}
\item компоновать (верстать) текст;
\item использовать всевозможные шрифты и полиграфические изображения;
\item осуществлять редактирование на уровне лучших текстовых процессоров;
\item обрабатывать графические изображения;
\item обеспечивать вывод документов высокого качества;\\
и др.
\end{itemize}

\subsection{Издательская система \TeX}
\textbf{\TeX} — система компьютерной вёрстки, разработанная американским профессором информатики Дональдом Кнутом в целях создания компьютерной типографии. В неё входят средства для секционирования документов, для работы с перекрёстными ссылками.

В отличие от обыкновенных текстовых процессоров и систем компьютерной вёрстки, построенных по принципу WYSIWYG, в \TeX ’е пользователь лишь задает текст и его структуру, а \TeX самостоятельно на основе выбранного пользователем шаблона форматирует документ, заменяя при этом дизайнера и верстальщика.
\subsection{Дональд Кнут}
\textbf{Дональд Эрвин Кнут} (\textit{род. 10 января 1938 года, Милуоки, штат Висконсин}) — американский учёный в области информатики, эмерит-профессор Стэнфордского университета, профессор СПбГУ и других университетов, преподаватель и идеолог программирования, автор 19 монографий (в том числе ряда классических книг по программированию) и более 160 статей, разработчик нескольких известных программных технологий. Автор всемирно известной серии книг, посвящённой основным алгоритмам и методам вычислительной математики, а также создатель настольных издательских систем \TeX  и METAFONT, предназначенных для набора и вёрстки книг научно-технической тематики (в первую очередь — физико-математических).
\subsection{Издательская система \LaTeX}
\textbf{\LaTeX} — наиболее популярный набор макрорасширений (или макропакет) системы компьютерной вёрстки \TeX , который облегчает набор сложных документов. В типографском наборе системы \TeX  форматируется традиционно как \LaTeX .

Важно заметить, что ни один из макропакетов для \TeX ’а не может расширить возможностей \TeX (всё, что можно сделать в \LaTeX ’е, можно сделать и в \TeX ’е без расширений), но, благодаря различным упрощениям, использование макропакетов зачастую позволяет избежать весьма изощрённого программирования.

Пакет позволяет автоматизировать многие задачи набора текста и подготовки статей, включая набор текста на нескольких языках, нумерацию разделов и формул, перекрёстные ссылки, размещение иллюстраций и таблиц на странице, ведение библиографии и др. Кроме базового набора существует множество пакетов расширения \LaTeX . Первая версия была выпущена Лесли Лэмпортом в 1984 году; текущая версия после создания в 1994 году испытывала некоторый период нестабильности, окончившийся к концу 1990-х годов, а в настоящее время стабилизировалась (хотя раз в год выходит новая версия).

Общий внешний вид документа в \LaTeX  определяется стилевым файлом. Существует несколько стандартных стилевых файлов для статей, книг, писем и т. д., кроме того, многие издательства и журналы предоставляют свои собственные стилевые файлы, что позволяет быстро оформить публикацию, соответствующую стандартам издания.

Термин \LaTeX относится только к языку разметки, он не является текстовым редактором. Для того, чтобы создать документ с его помощью, надо набрать .tex-файл с помощью какого-нибудь текстового редактора. В принципе, подойдёт любой редактор, но большая часть людей предпочитает использовать специализированные, которые так или иначе облегчают работу по набору текста \LaTeX -разметки.

Будучи распространяемым под лицензией \LaTeX  Project Public License, \LaTeX  относится к свободному программному обеспечению.
\subsection{Лесли Лэмпорт}
\textbf{Лесли Лэмпорт} (\textit{род. 7 февраля 1941 года, Нью-Йорк}) — американский учёный в области информатики, первый лауреат премии Дейкстры. Разработчик \LaTeX  — популярного набора макрорасширений системы компьютерной вёрстки \TeX, исследователь теории распределённых систем, темпоральной логики и вопросов синхронизации процессов во взаимодействующих системах. 

\begin{itemize}
\item Лауреат Премии Тьюринга (2013);
\item Член Национальной академии наук США (2011);
\item Член Национальной инженерной академии США (1991).
\end{itemize}
\section{Основные правила создания текстового документа}
Перед тем как перейти к работе с программой необходимо рассмотреть синтаксис и основные инструкции \LaTeX , чтобы вы могли чувствовать себя уверенно. Мы не будем рассматривать все команды \LaTeX , это слишком много, остановимся только на тех, которые будем использовать. Общий синтаксис команды:\\

\textbackslash имя команды[параметр1,параметр2] $\lbrace$аргумент1$\rbrace$ $\lbrace$аргумент2$\rbrace$\\

Имя инструкции чувствительно к регистру и она должна обязательно начинаться с косой черты. Некоторым командам передаются параметры, они настраивают их особенности работы, а в фигурных скобках передаются аргументы, это данные, с которыми будет работать команда. А теперь разберем команды:
\begin{itemize}
\item \textbackslash documentclass - описывает класс документа, статья, книга, отчет и так далее;
\item \textbackslash begin - указывает на начало тела документа или блока;
\item \textbackslash end - завершение документа или блока;
\item \textbackslash usepackage - загружает пакет команд \LaTeX  в текущий документ, нужно для настройки кодировки, шрифта и другого;
\item \textbackslash maketitle - создает титульный лист с названием и всем прочим;
\item \textbackslash tableofcontents - содержание статьи или книги;
\item \textbackslash chapter - создает главу;
\item \textbackslash section - создает раздел;
\item \textbackslash subsection - создает подраздел;
\item \textbackslash bfseries - жирный текст;
\item \textbackslash textit - курсив;
\item \textbackslash title - заголовок документа;
\item \textbackslash author - автор документа;
\item \textbackslash date - дата создания документа.
\end{itemize}

Сначала нужно создать новый файл \LaTeX . Для этого откройте меню "Файл" и выберите "New". В открывшемся диалоговом окне вам предстоит выбрать шаблон документа. Далее, нужно в поле title и author ввести название книги и имя автора, также тут можно указать дату, которая будет отображаться на титульном листе.Каждый документ \LaTeX  имеет определенную структуру, вначале идут настройки отображения, имортирование нужных пакетов, а уже потом сам текст в теле документа.
\end{document} % конец документа