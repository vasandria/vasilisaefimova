% "Лабораторная работа 8"

\documentclass[a4paper,12pt]{article} % тип документа

% report, book

%  Русский язык

\usepackage[T2A]{fontenc}			% кодировка
\usepackage[utf8]{inputenc}			% кодировка исходного текста
\usepackage[english,russian]{babel}	% локализация и переносы


% Математика
\usepackage{amsmath,amsfonts,amssymb,amsthm,mathtools} 


\usepackage{wasysym}
\usepackage{hyperref}


%Заговолок
\author{Ефимова В.С. ИВТ3}
\title{Тема 8. Инвариантная работа}
\date{\today}


\begin{document} % начало документа
\maketitle
\newpage
\begin{center}
\textbf{{\Huge МАТРИЦЫ}}
\begin{table}[h]
\begin{tabular}{|c c|c|c|}
\hline
$\backslash$ Команда & \{Окружение\} & Назначение & Примечание \\
\hline
$\backslash$ begin & \{matrix\} & создание матрицы без скобок & открывающая команда \\
$\backslash$ end & \{matrix\} &  & закрывающая команда \\
\hline
$\backslash$ begin & \{pmatrix\} & создание матрицы в круглых скобках & открывающая команда  \\
$\backslash$ end & \{pmatrix\} &  & закрывающая команда \\
\hline
$\backslash$ begin & \{bmatrix\} & создание матрицы в квадратных скобках & открывающая команда \\
$\backslash$ end & \{bmatrix\} &  & закрывающая команда \\
\hline
$\backslash$ begin & \{vmatrix\} & создание матрицы в прямых скобках & открывающая команда \\
$\backslash$ end & \{vmatrix\} &  & закрывающая команда \\
\hline
$\backslash$ begin & \{Vmatrix\} & создание матрицы в двойных прямых скобках & открывающая команда \\ 
$\backslash$ end & \{Vmatrix\} &  & закрывающая команда \\
\hline
\& &  & разделение между элементами матрицы&  \\
\hline
$\backslash$ $\backslash$ &  & новая строка &  \\
\hline
\end{tabular} 
\end{table}
\end{center}
\end{document}