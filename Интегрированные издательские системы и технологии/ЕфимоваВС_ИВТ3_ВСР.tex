% "Лабораторная работа 7"

\documentclass[a4paper,12pt]{article} % тип документа

% report, book

%  Русский язык

\usepackage[T2A]{fontenc}			% кодировка
\usepackage[utf8]{inputenc}			% кодировка исходного текста
\usepackage[english,russian]{babel}	% локализация и переносы


% Математика
\usepackage{amsmath,amsfonts,amssymb,amsthm,mathtools} 


\usepackage{wasysym}
\usepackage{hyperref}


%Заговолок
\author{Ефимова В.С. ИВТ3}
\title{Тема 7. Вариативная самостоятельная работа}
\date{\today}


\begin{document} % начало документа
\maketitle
\newpage
\section{Производная}
\textit{Определение}
Пусть функция $f(x)$ определена в некоторой окрестности точки
$x0$. Производной функции $f(x)$ в точке $x0$ называется
\begin{equation*}
\lim_{\Delta x \to 0}\frac{f(x_{0}+\Delta x)-f(x_{0})}{\Delta x}=\lim_{\Delta x \to 0}\frac{\Delta f}{\Delta x}.
\end{equation*}
Обозначение: $f'(x_{0}), \dfrac{df(x_{0})}{dx}.$
Если 
\begin{equation*}
\lim_{\Delta x \to 0}\frac{\Delta f)}{\Delta x}=\infty, +\infty, -\infty,
\end{equation*}
то говорят, что в точке $x0$ существует бесконечная производная.

\textit{Определение}
Правосторонней производной функции $f(x)$ в точке $x0$ называется
\begin{equation*}
\lim_{\Delta x \to 0+0}\frac{\Delta f)}{\Delta x}
\end{equation*}
Обозначение: $f'_{+}(x_{0})$

\textit{Определение}
Левосторонней производной функции $f(x)$ в точке $x0$ называется
\begin{equation*}
\lim_{\Delta x \to 0-0}\frac{\Delta f)}{\Delta x}
\end{equation*}
Обозначение: $f'_{-}(x_{0})$

\textit{Определение}
Правосторонняя и левосторонняя производные называются односторонними производными.

\textit{Теорема (о связи односторонних производных с двусторонней)}
$\exists f'(x_{0})=A\Leftrightarrow\exists f'_{+}(x_{0})=A, f'_(x_{0})=A.$

\textit{Определение}
Процесс нахождения производной называется дифференцированием.

\section{Физический смысл производной}
Пусть $S(t)$ - длина пути, пройденного телом за время $t$. Тогда
средняя скорость движения тела на интервале $[t, t + \Delta t]$ будет

\begin{equation*}
V_{sr}=\dfrac{S(t+\Delta t)- S(t)}{\Delta t}.
\end{equation*}
Соответственно, мгновенная скорость движения будет равна
\begin{equation*}
V\dfrac{dS}{dT}.
\end{equation*}

\section{Вычисление производных}

\textit{Производные основных элементарных функций:}
\begin{enumerate}
\item $c'=0$
\item $(x^{a})'=\alpha^{x}\ln a$
\item $(a^{x})'=a^{x}\ln a$
\item $(e^{x})'=e^{x}$
\item $(\log_{a}x)'=\dfrac{1}{x ln a}$
\item $(\ln x)'=\dfrac{1}{x}$
\item $(\sin x)'=\cos x$
\item $(\cos x)'=-\sin x$
\item $(\tg x)'=\dfrac{1}{\cos^{2}x}$
\item $(\ctg x)'=-\dfrac{1}{\sin^{2}x}$
\item $(\arcsin x)'=\dfrac{1}{\sqrt{1-x^{2}}}$
\item $(\arccos x)'=-\dfrac{1}{\sqrt{1-x^{2}}}$
\item $(\arctg x)'=\dfrac{1}{\sqrt{1+x^{2}}}$
\item $(\arcctg x)'=-\dfrac{1}{\sqrt{1+x^{2}}}$
\end{enumerate}

\textit{Правила нахождения производных, связанные с арифметическими
действиями над функциями:}
\begin{enumerate}
\item $(u+v)'=u'+v'$
\item $(uv)'=u'v+uv'$
\item $(\dfrac{u}{v})'=\dfrac{u'v-uv'}{v^{2}}$
\item $(cu)'=c \cdot u'$
\item $c'=0$
\end{enumerate}
\end{document} % конец документа